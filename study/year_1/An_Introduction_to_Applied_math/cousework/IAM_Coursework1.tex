%%%%%%%%%%%%%%%%%%%%%%%%%%%%%%%%%%%%%%%%%%%%%%%%%%%%%%%%%%%%%%%%%%
%%%%%%%%%%%%%%%%%%%%%%%%%%%%%%%%%%%%%%%%%%%%%%%%%%%%%%%%%%%%%%%%%%
%Packages
\documentclass[10pt, a4paper]{article}
\usepackage[top=3cm, bottom=4cm, left=3.5cm, right=3.5cm]{geometry}
\usepackage{amsmath,amsthm,amsfonts,amssymb,amscd, fancyhdr, color, comment, graphicx, environ, pifont}
\usepackage{float}
\usepackage{mathrsfs}
\usepackage[math-style=ISO]{unicode-math}
\setmathfont{TeX Gyre Termes Math}
\usepackage[dvipsnames]{xcolor}
\usepackage[framemethod=TikZ]{mdframed}
\usepackage{enumerate}
\usepackage[shortlabels]{enumitem}
\usepackage{fancyhdr}
\usepackage{indentfirst}
\usepackage{listings}
\usepackage{sectsty}
\usepackage{thmtools}
\usepackage{shadethm}
\usepackage{hyperref}
\usepackage{setspace}
\usepackage[linguistics]{forest}
\hypersetup{
    colorlinks=true,
    linkcolor=blue,
    filecolor=magenta,      
    urlcolor=blue,
}
%%%%%%%%%%%%%%%%%%%%%%%%%%%%%%%%%%%%%%%%%%%%%%%%%%%%%%%%%%%%%%%%%%
%%%%%%%%%%%%%%%%%%%%%%%%%%%%%%%%%%%%%%%%%%%%%%%%%%%%%%%%%%%%%%%%%%
%Environment setup
\mdfsetup{skipabove=\topskip,skipbelow=\topskip}
\newrobustcmd\ExampleText{%
An \textit{inhomogeneous linear} differential equation has the form
\begin{align}
L[v ] = f,
\end{align}
where $$L$$ is a linear differential operator, $$v$$ is the dependent
variable, and $$f$$ is a given non−zero function of the independent
variables alone.
}
\mdfdefinestyle{theoremstyle}{%
linecolor=black,linewidth=1pt,%
frametitlerule=true,%
frametitlebackgroundcolor=gray!20,
innertopmargin=\topskip,
}
\mdtheorem[style=theoremstyle]{Problem}{Problem}
\newenvironment{Solution}{\textbf{Solution.}}

\definecolor{codegreen}{rgb}{0,0.6,0}
\definecolor{codegray}{rgb}{0.5,0.5,0.5}
\definecolor{codepurple}{rgb}{0.58,0,0.82}
\definecolor{backcolour}{rgb}{0.95,0.95,0.92}

\lstdefinestyle{mystyle}{
    backgroundcolor=\color{backcolour},   
    commentstyle=\color{codegreen},
    keywordstyle=\color{magenta},
    numberstyle=\tiny\color{codegray},
    stringstyle=\color{codepurple},
    basicstyle=\ttfamily\footnotesize,
    breakatwhitespace=false,         
    breaklines=true,                 
    captionpos=b,                    
    keepspaces=true,                 
    numbers=left,                    
    numbersep=5pt,                  
    showspaces=false,                
    showstringspaces=false,
    showtabs=false,                  
    tabsize=2
}

\lstset{style=mystyle}
%%%%%%%%%%%%%%%%%%%%%%%%%%%%%%%%%%%%%%%%%%%%%%%%%%%%%%%%%%%%%%%%%%
%%%%%%%%%%%%%%%%%%%%%%%%%%%%%%%%%%%%%%%%%%%%%%%%%%%%%%%%%%%%%%%%%%
%Fill in the appropriate information below
\newcommand{\norm}[1]{\left\lVert#1\right\rVert}     
\newcommand\course{Course name}                            % <-- course name   
\newcommand\hwnumber{COMP400XXX-X}                                 % <-- homework number
\newcommand\Information{(Your name)}                        % <-- personal information
%%%%%%%%%%%%%%%%%%%%%%%%%%%%%%%%%%%%%%%%%%%%%%%%%%%%%%%%%%%%%%%%%%
%%%%%%%%%%%%%%%%%%%%%%%%%%%%%%%%%%%%%%%%%%%%%%%%%%%%%%%%%%%%%%%%%%
%Page setup
\pagestyle{fancy}
\headheight 35pt
\lhead{\today}
\rhead{\includegraphics[width=2.5cm]{icl_logo.png}}
\lfoot{}
\pagenumbering{arabic}
\cfoot{\small\thepage}
\rfoot{}
\headsep 1.2em
\renewcommand{\baselinestretch}{1.25}
%%%%%%%%%%%%%%%%%%%%%%%%%%%%%%%%%%%%%%%%%%%%%%%%%%%%%%%%%%%%%%%%%%
%%%%%%%%%%%%%%%%%%%%%%%%%%%%%%%%%%%%%%%%%%%%%%%%%%%%%%%%%%%%%%%%%%
%Add new commands here
\renewcommand{\labelenumi}{\alph{enumi})}
\newcommand{\Z}{\mathbb Z}
\newcommand{\R}{\mathbb R}
\newcommand{\Q}{\mathbb Q}
\newcommand{\NN}{\mathbb N}
\newcommand{\PP}{\mathbb P}
\DeclareMathOperator{\Mod}{Mod} 
\renewcommand\lstlistingname{Algorithm}
\renewcommand\lstlistlistingname{Algorithms}
\def\lstlistingautorefname{Alg.}
\newtheorem*{theorem}{Theorem}
\newtheorem*{lemma}{Lemma}
\newtheorem{case}{Case}
\newcommand{\assign}{:=}
\newcommand{\infixiff}{\text{ iff }}
\newcommand{\nobracket}{}
\newcommand{\backassign}{=:}
\newcommand{\tmmathbf}[1]{\ensuremath{\boldsymbol{#1}}}
\newcommand{\tmop}[1]{\ensuremath{\operatorname{#1}}}
\newcommand{\tmtextbf}[1]{\text{{\bfseries{#1}}}}
\newcommand{\tmtextit}[1]{\text{{\itshape{#1}}}}

\newenvironment{itemizedot}{\begin{itemize} \renewcommand{\labelitemi}{$$\bullet$$}\renewcommand{\labelitemii}{$$\bullet$$}\renewcommand{\labelitemiii}{$$\bullet$$}\renewcommand{\labelitemiv}{$$\bullet$$}}{\end{itemize}}
\catcode`\<=\active \def<{
\fontencoding{T1}\selectfont\symbol{60}\fontencoding{\encodingdefault}}
\catcode`\>=\active \def>{
\fontencoding{T1}\selectfont\symbol{62}\fontencoding{\encodingdefault}}
\catcode`\<=\active \def<{
\fontencoding{T1}\selectfont\symbol{60}\fontencoding{\encodingdefault}}

%%%%%%%%%%%%%%%%%%%%%%%%%%%%%%%%%%%%%%%%%%%%%%%%%%%%%%%%%%%%%%%%%%
%%%%%%%%%%%%%%%%%%%%%%%%%%%%%%%%%%%%%%%%%%%%%%%%%%%%%%%%%%%%%%%%%%
%Begin now!



\begin{document}

\begin{titlepage}
    \begin{center}
        \vspace*{3cm}
            
        \Huge
        \textbf{
        Coursework I}
            
            
        \vspace{1.5cm}
        \Large
            
        \textbf{
        CID number: A mysterious number series}% <-- author
        
            
        \vfill
        
    MATH40007: Introduction to Applied Mathematics, 2023
        \vspace{1cm}
            
        \includegraphics[width=0.4\textwidth]{icl_logo.png}
        \\
        
        \Large
        
        \today
            
    \end{center}
\end{titlepage}


\newpage
\begin{Problem}
Coursework background: Let $$N>1$$ be a positive integer. Given an $$N$$-node graph viewed as a resistive electric circuit with all edges having unit conductance, the resistance matrix $$\mathbf{R}$$ is defined as the matrix with components $$R_{i j}$$ given by the effective resistance between node $$i$$ (set to unit voltage) and node $$j$$ (grounded) with all diagonal elements taken to vanish, i.e. $$R_{i i}=0$$. Recall that the effective resistance is just the reciprocal of the effective conductance $$C_{\text {eff }}$$ introduced in lectures.

The total effective resistance of the graph is defined in terms of this resistance matrix as

$$$$
R^{(total)}=\sum_{i<j} R_{i j} .
$$$$

Let the orthonormal (i.e., unit length and mutually orthogonal) eigenvectors and corresponding eigenvalues of the $$N-$$ by $$-N$$ graph Laplacian $$\mathbf{K}$$ be denoted by

$$$$
\left\{\mathbf{e}_{k} \mid k=1, \ldots, N\right\}, \quad\left\{\lambda_{k} \mid k=1, \ldots, N\right\},
$$$$
where we take

$$$$
\mathbf{e}_{1}=\frac{1}{\sqrt{N}}\left(\begin{array}{c}
1 \\
1 \\
. \\
\cdot \\
1
\end{array}\right), \quad \lambda_{1}=0
$$$$

and where we denote the elements of $$\mathbf{e}_{k}$$ for $$k=2, \ldots, N$$ as

$$$$
\mathbf{e}_{k}=\left(\begin{array}{r}
e_{k 1} \\
e_{k 2} \\
\cdot \\
\cdot \\
e_{k N}
\end{array}\right), \quad k=2, \ldots, N
$$$$

The following two formulas can be established relating the elements of the resistance matrix and the total effective resistance to the eigenvectors/eigenvalues of K: 
\begin{equation*}
R_{i j}=\sum_{k=2}^{N} \frac{1}{\lambda_{k}}\left(e_{k i}-e_{k j}\right)^{2}
\end{equation*}

and

\begin{equation*}
R^{(total)}=N \sum_{k=2}^{N} \frac{1}{\lambda_{k}} .
\end{equation*}
Coursework exercise: Consider the 4-node graph shown in Figure 1:
\begin{center}
\includegraphics[scale=0.25]{2023_02_04_3a1caedc5be036335f24g-2.jpg}
\end{center}
\begin{center}
Figure 1: A 4-node graph viewed as an electric circuit.
\end{center}
\begin{itemize}
\item[(a)] Using the node labelling given in the figure, find the resistance matrix $$\mathbf{R}$$ for this graph.

\item[(b)] Hence, using your results from part (a), compute $$R^{\text {total }}$$ for this graph using formula (1).

\item[(c)] Find the eigenvalues $$\left\{\lambda_{k} \mid k=1, \ldots, 4\right\}$$ and orthonormal eigenvectors $$\left\{\mathbf{e}_{k} \mid k=\right.$$ $$1, \ldots, 4\}$$ of the graph Laplacian $$\mathbf{K}$$.

\item[(d)] Use the results of parts (a) and (c) to verify formula (2).

\item[(e)] Use the results of parts (b) and (c) to verify formula (3).

\item[(f)] Prove formula (2). Hint: consider using the eigenvectors as a basis.

[Note: you are not required to prove formula (3)].
\end{itemize}
\end{Problem}
\begin{Solution}
    \begin{itemize}
        \item [(a)]
        As each node in the graph is connected to three edges, the degree matrix $$\mathbf{D}$$ of the graph is:
        \begin{center}
        $$\mathbf{D}=$$
        \begin{pmatrix}
        3&0&0&0\\
        0&3&0&0\\
        0&0&3&0\\
        0&0&0&3
        \end{pmatrix}
        \end{center}

    Since all the nodes in the graph are connected with each other, the adjacent matrix $$\mathbf{W}$$ of the graph is:
    \begin{center}
        $$\mathbf{W}=$$
        \begin{pmatrix}
            0&1&1&1\\
            1&0&1&1\\
            1&1&0&1\\
            1&1&1&0
        \end{pmatrix}
    \end{center}
    since all the nodes in the graph are connected with each other.

    Therefore, the Laplacian matrix $$\mathbf{K}$$ of the graph is:
    \begin{center}
        $$\mathbf{K}=\mathbf{D}-\mathbf{W}=$$
    \begin{pmatrix}
        3&-1&-1&-1\\
        -1&3&-1&-1\\
        -1&-1&3&-1\\
        -1&-1&-1&3
    \end{pmatrix}
    \end{center}
    Since the graph is complete, it followes that the effective conductances between every pair of two nodes are the same. So, we could only caculate the conductance of one pair of the $$4$$ nodes. We can assume that the voltage of node \textcircled{\raisebox{-0.7pt}{1}} is 
    $$1$$ and the node \textcircled{\raisebox{-0.7pt}{4}} is grounded. Then, set the effective conductance between node \textcircled{\raisebox{-0.7pt}{1}} and node \textcircled{\raisebox{-0.7pt}{4}} is $$\mathbf{C_{eff}}$$. We have this
    \begin{equation*}
        \mathbf{K}\mathbf{x}=\mathbf{f}
    \end{equation*}
    We suppose the potentials at \textcircled{\raisebox{-0.7pt}{2}} and \textcircled{\raisebox{-0.7pt}{3}} are $$x_1$$ and $$x_2$$ respectvely.
    As KCL holds at node \textcircled{\raisebox{-0.7pt}{2}} and \textcircled{\raisebox{-0.7pt}{3}}, it follows that
    \begin{equation*}
        \begin{pmatrix}
            3&-1&-1&-1\\
        -1&3&-1&-1\\
        -1&-1&3&-1\\
        -1&-1&-1&3
        \end{pmatrix}
        \begin{pmatrix}
            1\\x_2\\x_3\\0
        \end{pmatrix}
        =
        \begin{pmatrix}
            \mathbf{C_{eff}}\\
            0\\
            0\\
            -\mathbf{C_{eff}}
        \end{pmatrix}
    \end{equation*}
    Because the nodes \textcircled{\raisebox{-0.7pt}{4}} is grounded, we can cancel the fouth row and the fourth column of the Laplacian matrix. Then, the grounded Laplacian matrix, with node \textcircled{\raisebox{-0.7pt}{4}} grounded, gives the linear system
    \begin{equation*}
        \hat{\mathbf{K}}\hat{\mathbf{x}}=
        \begin{pmatrix}
            3&-1&-1\\
            -1&3&-1\\
            -1&-1&3\\
        \end{pmatrix}
        \begin{pmatrix}
            1\\x_2\\x_3
        \end{pmatrix}
        =
        \begin{pmatrix}
            \mathbf{C_{eff}}\\
            0\\
            0\\
        \end{pmatrix}
    \end{equation*}

    \begin{equation*}
        \left\{
            \begin{array}{c}
                3-x_2-x_3=\mathbf{C_{eff}}\\
                -1+3x_2-x_3=0\\
                -1-x_2+3x_3=0
            \end{array}
            \Longrightarrow 
            \left\{
            \begin{array}{c}
                \mathbf{C_{eff}}=2\\
                x_2=\frac{1}{2}\\
                x_3=\frac{1}{2}
            \end{array}
    \end{equation*}
    Therefore, the effective conductance between each pair of the nodes is $$2$$. As the effective resistance is the reciprocal of the effective conductance, it follows that
    \begin{equation*}
        \mathbf{R_{eff}}=\frac{1}{\mathbf{C_{eff}}}=\frac{1}{2}.
    \end{equation*}
    The graph is complete,so the effective resistances between every pair of two nodes are the same. 
    Hence, by the definition of the resistance matrix $$\mathbf{R}$$ for this graph is
    \begin{equation*}
        \mathbf{R}=
        \begin{pmatrix}
            0&\frac{1}{2}&\frac{1}{2}&\frac{1}{2}\\
            \frac{1}{2}&0&\frac{1}{2}&\frac{1}{2}\\
            \frac{1}{2}&\frac{1}{2}&0&\frac{1}{2}\\
            \frac{1}{2}&\frac{1}{2}&\frac{1}{2}&0
        \end{pmatrix}
    \end{equation*}
    \item[(b)]
    $$$$
    R^{(total)}=\sum_{i<j} R_{i j}=\frac{1}{2}+\frac{1}{2}+\frac{1}{2}+\frac{1}{2}+\frac{1}{2}+\frac{1}{2}=\frac{1}{2}\times 6=3
    $$$$
    \item[(c)]
    The graph Laplacian matrix $$\mathbf{K}$$ is
    \begin{pmatrix}
        3&-1&-1&-1\\
        -1&3&-1&-1\\
        -1&-1&3&-1\\
        -1&-1&-1&3 
    \end{pmatrix}

    so,the characteristic polynomial of $$\mathbf{K}$$ is
    \begin{equation*}
        \det(\lambda I_n-\mathbf{K})=
        \begin{vmatrix}
        \lambda-3&1&1&1\\
        1&\lambda-3&1&1\\
        1&1&\lambda-3&1\\
        1&1&1&\lambda-3 
        \end{vmatrix}
        =\lambda(\lambda-4)^3.
    \end{equation*}
    $$$$
    \lambda(\lambda-4)^3=0\Longrightarrow \lambda=0\;or\;4.
    $$$$
    Hence, we have $$\lambda_1=0,\:\lambda_2=4,\:\lambda_3=4,\:\lambda_4=4$$.
    $$$$
    \{\lambda_k|k=1,\ldots,4\}=\{\lambda_1=0,\lambda_2=4,\lambda_3=4,\lambda_4=4\}
    $$$$
    When $$\lambda_1=0$$, the corresponding eigenvector is
    $$$$
        e_1=\frac{1}{\sqrt{4}}\begin{pmatrix}
            1\\1\\1\\1
        \end{pmatrix}
        =\frac{1}{2}\begin{pmatrix}
            1\\1\\1\\1
        \end{pmatrix}
    $$$$
    When $$\lambda_2=\lambda_3=\lambda_4=4$$, we have
    \begin{align*}
        \mathbf{K}\mathbf{x}&=4\mathbf{x}\\
        \mathbf{K}\mathbf{x}-4\mathbf{x}&=0\\
        (\mathbf{K}-4\mathbf{I})\mathbf{x}&=0\\
        \begin{pmatrix}
            -1&-1&-1&-1\\
            -1&-1&-1&-1\\
            -1&-1&-1&-1\\
            -1&-1&-1&-1\\
        \end{pmatrix}
        \begin{pmatrix}
            x_1\\x_2\\x_3\\x_4
        \end{pmatrix}
        &=0\\
        \begin{pmatrix}
            1&1&1&1\\
            1&1&1&1\\
            1&1&1&1\\
            1&1&1&1\\
        \end{pmatrix}
        \begin{pmatrix}
            x_1\\x_2\\x_3\\x_4
        \end{pmatrix}
        &=0\\
        x_1+x_2+x_3+x_4&=0
    \end{align*}
    Therefore, the eigenspace $$E_4$$ of $$\mathbf{K}$$ is the right null space of the matrix\\
    \begin{pmatrix}
        1&1&1&1\\
        1&1&1&1\\
        1&1&1&1\\
        1&1&1&1\\
    \end{pmatrix}
    It is easy to konw that the rank of this matrix is $$1$$ by elementary row operations and this is a $$4\times 4$$ matrix. Hence, by the rank nulity theorem, the dimension of the eigenspace of the matrix is $$3$$. Therefore,
    \begin{equation*}
        E_4= span \left\{
            \begin{pmatrix}
                -1\\1\\0\\0
            \end{pmatrix},
            \begin{pmatrix}
                -1\\0\\1\\0
            \end{pmatrix}
            \begin{pmatrix}
                -1\\0\\0\\1
            \end{pmatrix}
            \right\}
    \end{equation*}
    By some caculation, we can find one collection of the orthogonal eigenvectors are:
    \begin{align*}
        e_1&=\frac{1}{2}\begin{pmatrix}
            1\\1\\1\\1
        \end{pmatrix}
        \\
        e_2&=\frac{1}{\sqrt{2}}\begin{pmatrix}
           -1\\1\\0\\0
        \end{pmatrix}
        \\
        e_3&=\begin{pmatrix}
            -\frac{1}{\sqrt{6}}\\-\frac{1}{\sqrt{6}}\\ \frac{\sqrt{6}}{3}\\0
        \end{pmatrix}
        \\
        e_4&=\begin{pmatrix}
            -\frac{1}{2\sqrt{3}}\\-\frac{1}{2\sqrt{3}}\\-\frac{1}{2\sqrt{3}}\\ \frac{\sqrt{3}}{2}
        \end{pmatrix}
    \end{align*}
        It is easy to check their length is unit and they are mutually orthogonal. Hence,
        \begin{equation*}
            \{e_k|k=1,\ldots,4\}=\{e_1,e_2,e_3,e_4\}=\left\{\frac{1}{2}\begin{pmatrix}
                1\\1\\1\\1
            \end{pmatrix}, \frac{1}{\sqrt{2}}\begin{pmatrix}
            -1\\1\\0\\0\end{pmatrix},
            \begin{pmatrix}
                -\frac{1}{\sqrt{6}}\\-\frac{1}{\sqrt{6}}\\ \frac{\sqrt{6}}{3}\\0
            \end{pmatrix},
            \begin{pmatrix}
                -\frac{1}{2\sqrt{3}}\\-\frac{1}{2\sqrt{3}}\\-\frac{1}{2\sqrt{3}}\\ \frac{\sqrt{3}}{2}
            \end{pmatrix}
            \right\}
        \end{equation*}
    \item[(d)] The formula $$(2)$$ is
    $$$$
    R_{i j}=\sum_{k=2}^{N} \frac{1}{\lambda_{k}}\left(e_{k i}-e_{k j}\right)^{2}
    $$$$
    As $$\lambda_2=\lambda_3=\lambda_4=4$$ and $$e_1&=\frac{1}{2}\begin{pmatrix}
        1\\1\\1\\1
    \end{pmatrix}
    e_2&=\frac{1}{\sqrt{2}}\begin{pmatrix}
       -1\\1\\0\\0
    \end{pmatrix}
    e_3&=\begin{pmatrix}
        -\frac{1}{\sqrt{6}}\\-\frac{1}{\sqrt{6}}\\ \frac{\sqrt{6}}{3}\\0
    \end{pmatrix}
    e_4&=\begin{pmatrix}
        -\frac{1}{2\sqrt{3}}\\-\frac{1}{2\sqrt{3}}\\-\frac{1}{2\sqrt{3}}\\ \frac{\sqrt{3}}{2}
    \end{pmatrix}$$
    \begin{equation*}
        $$\mathbf{R}=$$\\
    \begin{pmatrix}
        0&\frac{1}{4}(\frac{1}{\sqrt{2}}+\frac{1}{\sqrt{2}})^2&\frac{1}{4}[\frac{1}{2}+(\frac{\sqrt{6}}{3}+\frac{1}{\sqrt6})^2]&\frac{1}{4}[\frac{1}{2}+\frac{1}{6}+(\frac{\sqrt{3}}{2}+\frac{1}{2\sqrt{3}})^2}]\\
        \frac{1}{4}(\frac{1}{\sqrt{2}}+\frac{1}{\sqrt{2}})^2&0&\frac{1}{4}[\frac{1}{2}+(\frac{\sqrt{6}}{3}+\frac{1}{\sqrt6})^2]&\frac{1}{4}[\frac{1}{2}+\frac{1}{6}+(\frac{\sqrt{3}}{2}+\frac{1}{2\sqrt{3}})^2}]\\
        \frac{1}{4}[\frac{1}{2}+(\frac{\sqrt{6}}{3}+\frac{1}{\sqrt6})^2]&\frac{1}{4}[\frac{1}{2}+(\frac{\sqrt{6}}{3}+\frac{1}{\sqrt6})^2]&0&\frac{1}{4}[\frac{1}{2}+\frac{1}{6}+(\frac{\sqrt{3}}{2}+\frac{1}{2\sqrt{3}})^2}]\\
        \frac{1}{4}[\frac{1}{2}+\frac{1}{6}+(\frac{\sqrt{3}}{2}+\frac{1}{2\sqrt{3}})^2}]&\frac{1}{4}[\frac{1}{2}+\frac{1}{6}+(\frac{\sqrt{3}}{2}+\frac{1}{2\sqrt{3}})^2}]&\frac{1}{4}[\frac{1}{2}+\frac{1}{6}+(\frac{\sqrt{3}}{2}+\frac{1}{2\sqrt{3}})^2}]&0
    \end{pmatrix}
    \\
    = \begin{pmatrix}
        0&\frac{1}{2}&\frac{1}{2}&\frac{1}{2}\\
        \frac{1}{2}&0&\frac{1}{2}&\frac{1}{2}\\
        \frac{1}{2}&\frac{1}{2}&0&\frac{1}{2}\\
        \frac{1}{2}&\frac{1}{2}&\frac{1}{2}&0
    \end{pmatrix}
    \end{equation*}\\
    which is the same as the result in part (a). Therefore, the formula $$(2)$$ is correct.
    \item[(e)]
    The formula $$(3)$$ is 
    \begin{equation*}
    R^{(total)}=N \sum_{k=2}^{N} \frac{1}{\lambda_{k}}    
    \end{equation*}
    Since $$N=4$$ and $$\lambda_2=\lambda_3=\lambda_3=4$$,
    \begin{equation*}
        R^{(total)}=N \sum_{k=2}^{N} \frac{1}{\lambda_{k}}=4\times(\frac{1}{4}+\frac{1}{4}+\frac{1}{4})=3.
    \end{equation*}
    which is the same as the result in part (b). Therefore, the formula $$(3)$$ is correct.
    \item[(f)]
    \begin{proof}
        We use the eigenvectors $$E=\{e_1,e_2,e_3,e_4\}$$ as a basis of $$\mathbb{R}^4$$. Regard $$\mathbf{K}$$ as a linear transformation,
        \begin{equation*}
            \mathbf{K}:\mathbb{R}^4 \to \mathbb{R}^4
        \end{equation*}
        Set the standard basis of $$\mathbb{R^4}$$ is $$B$$.
        From the Linear Algebra and Groups course, we have basis change fomula, where
        \begin{equation*}
            _{B}[id]_E=\begin{pmatrix}
                e_{11}&e_{21}&e_{31}&e_{41}\\
                e_{12}&e_{22}&e_{32}&e_{42}\\
                e_{13}&e_{23}&e_{33}&e_{43}\\
                e_{14}&e_{24}&e_{34}&e_{44}
            \end{pmatrix}
        \end{equation*}
        We set $$_{B}[id]_E$$ as $$P$$.
        As the matrix $$_{B}[id]_E$$ is consist of four orthogonal unit vectors, the inverse of it is 
        \begin{equation*}
            _{E}[id]_B=(_{B}[id]_E)^T=\begin{pmatrix}
                e_{11}&e_{12}&e_{13}&e_{14}\\
                e_{21}&e_{22}&e_{23}&e_{24}\\
                e_{31}&e_{32}&e_{33}&e_{34}\\
                e_{41}&e_{42}&e_{43}&e_{44}
            \end{pmatrix}
        \end{equation*}
        By the basis change formula, we have
        \begin{equation*}
            [\mathbf{K}]_E=P^{-1}\mathbf{K}P
        \end{equation*}
        As the column of $$P$$ is consist of eigenvectors of $$\mathbf{K}$$, $$[\mathbf{K}]_E$$ is a diagonal matrix whose element on the diagonal is the eigenvalues of $$\mathbf{K}$$, we set it as $$D$$, where
        \begin{equation*}
            D=\begin{pmatrix}
                \lambda_1&0&0&0\\
                0&\lambda_2&0&0\\
                0&0&\lambda_3&0\\
                0&0&0&\lambda_4
            \end{pmatrix}
        \end{equation*}
        Hence,
        \begin{align*}
            D&=P^{-1}\mathbf{K}P\\
            \mathbf{K}&=PDP^{-1}\\
            \mathbf{K}&=\begin{pmatrix}
                e_{11}&e_{21}&e_{31}&e_{41}\\
                e_{12}&e_{22}&e_{32}&e_{42}\\
                e_{13}&e_{23}&e_{33}&e_{43}\\
                e_{14}&e_{24}&e_{34}&e_{44}
            \end{pmatrix}
            \begin{pmatrix}
                \lambda_1&0&0&0\\
                0&\lambda_2&0&0\\
                0&0&\lambda_3&0\\
                0&0&0&\lambda_4
            \end{pmatrix}
            \begin{pmatrix}
                e_{11}&e_{12}&e_{13}&e_{14}\\
                e_{21}&e_{22}&e_{23}&e_{24}\\
                e_{31}&e_{32}&e_{33}&e_{34}\\
                e_{41}&e_{42}&e_{43}&e_{44}
            \end{pmatrix}\\
            &=\begin{pmatrix}
                e_1&e_2&e_3&e_4
            \end{pmatrix}
            \begin{pmatrix}
                \lambda_1&0&0&0\\
                0&\lambda_2&0&0\\
                0&0&\lambda_3&0\\
                0&0&0&\lambda_4
            \end{pmatrix}
            \begin{pmatrix}
                e_1^T\\e_2^T\\e_3^T\\e_4^T
            \end{pmatrix}
        \end{align*}
        Use the result of part (c),
        \begin{equation*}\tag{\ast }
            \mathbf{K}=\begin{pmatrix}
                1&e_{21}&e_{31}&e_{41}\\
                1&e_{22}&e_{32}&e_{42}\\
                1&e_{23}&e_{33}&e_{43}\\
                1&e_{24}&e_{34}&e_{44}
            \end{pmatrix}
            \begin{pmatrix}
                0&0&0&0\\
                0&4&0&0\\
                0&0&4&0\\
                0&0&0&4
            \end{pmatrix}
            \begin{pmatrix}
                1&1&1&1\\
                e_{21}&e_{22}&e_{23}&e_{24}\\
                e_{31}&e_{32}&e_{33}&e_{34}\\
                e_{41}&e_{42}&e_{43}&e_{44}
            \end{pmatrix}
        \end{equation*}
        Then, use the way in part (a) that we calculate $$R_{i j}$$ and substitue $$\mathbf{K}$$ with $$(\ast)$$, we can immediately get
        \begin{equation*}
            R_{i j}=\sum_{k=2}^N\frac{1}{\lambda_k}(e_{ki}-e_{kj})^2.
        \end{equation*}
    \end{proof}
\end{itemize}
  
\end{Solution}
%%%%%%%%%%%%%%%%%%%%%%%%%%%%%%%%%%%%%%%%%%%%%%%%%%%%%%%%%%%%%%%%%%
%Complete the assignment now
\end{document}
%%%%%%%%%%%%%%%%%%%%%%%%%%%%%%%%%%%%%%%%%%%%%%%%%%%%%%%%%%%%%%%%%%
%%%%%%%%%%%%%%%%%%%%%%%%%%%%%%%%%%%%%%%%%%%%%%%%%%%%%%%%%%%%%%%%%%