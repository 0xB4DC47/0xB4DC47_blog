\documentclass[10pt, a4paper]{article}
\usepackage[top=3cm, bottom=4cm, left=3.5cm, right=3.5cm]{geometry}
\usepackage{amsmath,amsthm,amsfonts,amssymb,amscd, fancyhdr, color, comment, graphicx, environ, pifont}
\usepackage{float}
\usepackage{mathrsfs}
\usepackage[math-style=ISO]{unicode-math}
\usepackage[framemethod=TikZ]{mdframed}
\usepackage{enumerate}
\usepackage[shortlabels]{enumitem}
\usepackage{fancyhdr}
\usepackage{indentfirst}
\usepackage{listings}
\usepackage{sectsty}
\usepackage{thmtools}
\usepackage{shadethm}
\usepackage{hyperref}
\usepackage{setspace}
\usepackage[linguistics]{forest}
\hypersetup{
    colorlinks=true,
    linkcolor=blue,
    filecolor=magenta,      
    urlcolor=blue,
}
\mdfsetup{skipabove=\topskip,skipbelow=\topskip}
\mdfdefinestyle{theoremstyle}{%
linecolor=black,linewidth=1pt,%
frametitlerule=true,%
frametitlebackgroundcolor=gray!20,
innertopmargin=\topskip,
}
\mdtheorem[style=theoremstyle]{Problem}{Problem}
\newenvironment{Solution}{\textbf{Solution.}}

\definecolor{codegreen}{rgb}{0,0.6,0}
\definecolor{codegray}{rgb}{0.5,0.5,0.5}
\definecolor{codepurple}{rgb}{0.58,0,0.82}
\definecolor{backcolour}{rgb}{0.95,0.95,0.92}

\lstdefinestyle{mystyle}{
    backgroundcolor=\color{backcolour},   
    commentstyle=\color{codegreen},
    keywordstyle=\color{magenta},
    numberstyle=\tiny\color{codegray},
    stringstyle=\color{codepurple},
    basicstyle=\ttfamily\footnotesize,
    breakatwhitespace=false,         
    breaklines=true,                 
    captionpos=b,                    
    keepspaces=true,                 
    numbers=left,                    
    numbersep=5pt,                  
    showspaces=false,                
    showstringspaces=false,
    showtabs=false,                  
    tabsize=2
}

\lstset{style=mystyle}
\newcommand{\norm}[1]{\left\lVert#1\right\rVert}     
\newcommand\course{Analysis I}                            
\newcommand\hwnumber{MATH40002}                                  
\pagestyle{fancy}
\headheight 35pt
\lhead{\today}
\rhead{\includegraphics[width=2.5cm]{icl_logo.png}}
\lfoot{}
\pagenumbering{arabic}
\cfoot{\small\thepage}
\rfoot{}
\headsep 1.2em
\renewcommand{\baselinestretch}{1.25}
\renewcommand{\labelenumi}{\alph{enumi}}
\newcommand{\Z}{\mathbb Z}
\newcommand{\R}{\mathbb R}
\newcommand{\Q}{\mathbb Q}
\newcommand{\NN}{\mathbb N}
\newcommand{\PP}{\mathbb P}
\DeclareMathOperator{\Mod}{Mod} 
\renewcommand\lstlistingname{Algorithm}
\renewcommand\lstlistlistingname{Algorithms}
\def\lstlistingautorefname{Alg.}
\newtheorem*{theorem}{Theorem}
\newtheorem*{lemma}{Lemma}
\newtheorem{case}{Case}
\newcommand{\assign}{:=}
\newcommand{\infixiff}{\text{ iff }}
\newcommand{\nobracket}{}
\newcommand{\backassign}{=:}
\newcommand{\tmmathbf}[1]{\ensuremath{\boldsymbol{#1}}}
\newcommand{\tmop}[1]{\ensuremath{\operatorname{#1}}}
\newcommand{\tmtextbf}[1]{\text{{\bfseries{#1}}}}
\newcommand{\tmtextit}[1]{\text{{\itshape{#1}}}}

\newenvironment{itemizedot}{\begin{itemize} \renewcommand{\labelitemi}{$$\bullet$$}\renewcommand{\labelitemii}{$$\bullet$$}\renewcommand{\labelitemiii}{$$\bullet$$}\renewcommand{\labelitemiv}{$$\bullet$$}}{\end{itemize}}
\catcode`\<=\active \def<{
\fontencoding{T1}\selectfont\symbol{60}\fontencoding{\encodingdefault}}
\catcode`\>=\active \def>{
\fontencoding{T1}\selectfont\symbol{62}\fontencoding{\encodingdefault}}
\catcode`\<=\active \def<{
\fontencoding{T1}\selectfont\symbol{60}\fontencoding{\encodingdefault}}
\begin{document}

\begin{titlepage}
    \begin{center}
        \vspace*{3cm}
            
        \Huge
        \textbf{
        Coursework I}
            
            
        \vspace{1.5cm}
        \Large
            
        \textbf{
        CID number: This number does not exist!}% <-- author
        
            
        \vfill
        
MATH40002: Analysis I
        \vspace{1cm}
            
        \includegraphics[width=0.4\textwidth]{icl_logo.png}
        \\
        
        \Large
        
        \today
            
    \end{center}
\end{titlepage}


\newpage
\begin{Problem}
    Prove that if $$f$$ and $$g$$ are continuous functions such that $$f(q)=g(q)$$ for all $$q \in \mathbb{Q}$$, then $$f=g$$.
\end{Problem}
\begin{Solution}
    \begin{proof}
        We prove this by using the sequential continuity. It states that if $$f$$ is continuous at $$x$$, then for any sequence $$\lim _{n \rightarrow \infty} x_{n}=x$$, we have $$\lim _{n \rightarrow \infty} f\left(x_{n}\right)=f(x)$$.

        Let $$x$$ be any real number. Since $$\mathbb{Q}$$ is dense in $$\mathbb{R}$$, we can always find a sequence of rational numbers $$\left\{q_{n}\right\} \subset \mathbb{Q}$$, which satisfies,
        
        $$$$
        \lim _{n \rightarrow \infty} q_{n}=x
        $$$$
        
        for any real number $$x$$.
        
        As $$f(q)=g(q)$$ for all $$q \in \mathbb{Q}$$, we have $$f\left(q_{n}\right)=g\left(q_{n}\right)$$ for any $$n \in \mathbb{N}$$. It follows that
        
        $$$$
        \lim _{n \rightarrow \infty} f\left(q_{n}\right)=\lim _{n \rightarrow \infty} g\left(q_{n}\right)
        $$$$
        
        Since $$f$$ and $$g$$ are continuous functions, then they are continuous at any $$x \in \mathbb{R}$$ by the definition of continuous function. Then, by the sequential continuity we stated before and the equation (1), we have
        
        $$$$
        \begin{aligned}
        & \lim _{n \rightarrow \infty} f\left(q_{n}\right)=f(x) . \\
        & \lim _{n \rightarrow \infty} g\left(q_{n}\right)=g(x) .
        \end{aligned}
        $$$$
        
        By the equations (2), (3), (4), we immediately get this:
        
        $$$$
        f(x)=g(x) \forall x \in \mathbb{R}
        $$$$
        
        It means that $$f=g$$.
    \end{proof}
    \begin{Problem}
        Prove that the finite union of bounded sets is bounded.
    \end{Problem}
    \begin{Solution}
       \begin{proof}
        Suppose we have a collection of finite bounded sets, which is $$\left\{S_{1}, S_{2}, S_{3}, \ldots, S_{n}\right\}$$. Each bounded set has its own upper bound, which is $$\left\{M_{1}, M_{2}, M_{3}, \ldots, M_{n}\right\}$$ respectively and has its own lower bound, which is $$\left\{m_{1}, m_{2}, m_{3}, \ldots, m_{n}\right\}$$ respectively. We claim that the finite union of these sets $$U_{i=1}^{n} S_{i}$$ is upper bounded by $$\max \left\{M_{1}, M_{2}, M_{3}, \ldots, M_{n}\right\}$$ and lower bounded by $$\min \left\{m_{1}, m_{2}, m_{3} \ldots, m_{n}\right\}$$.

Since any element $$x \in U_{i=1}^{n} S_{i}$$ must be in at least one of the bounded sets $$\left\{S_{1}, S_{2}, S_{3}, \ldots, S_{n}\right\}$$, then for any $$x \in U_{i=1}^{n} S_{i}$$, we have $$x \leq \max \left\{M_{1}, M_{2}, M_{3}, \ldots, M_{n}\right\}$$ and $$x \geq \min \left\{m_{1}, m_{2}, m_{3} \ldots, m_{n}\right\}$$. Therefore, the finite union of bounded sets is also bounded.
        \end{proof}
    \end{Solution}
\end{Solution}
\end{document}