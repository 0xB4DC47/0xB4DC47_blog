\documentclass[10pt, a4paper]{article}
\usepackage[top=3cm, bottom=4cm, left=3.5cm, right=3.5cm]{geometry}
\usepackage{amsmath,amsthm,amsfonts,amssymb,amscd, fancyhdr, color, comment, graphicx, environ, pifont}
\usepackage{float}
\usepackage{mathrsfs}
\usepackage[math-style=ISO]{unicode-math}
\usepackage[framemethod=TikZ]{mdframed}
\usepackage{enumerate}
\usepackage[shortlabels]{enumitem}
\usepackage{fancyhdr}
\usepackage{indentfirst}
\usepackage{listings}
\usepackage{sectsty}
\usepackage{thmtools}
\usepackage{shadethm}
\usepackage{hyperref}
\usepackage{setspace}
\usepackage[linguistics]{forest}
\hypersetup{
    colorlinks=true,
    linkcolor=blue,
    filecolor=magenta,      
    urlcolor=blue,
}
\mdfsetup{skipabove=\topskip,skipbelow=\topskip}
\mdfdefinestyle{theoremstyle}{%
linecolor=black,linewidth=1pt,%
frametitlerule=true,%
frametitlebackgroundcolor=gray!20,
innertopmargin=\topskip,
}
\mdtheorem[style=theoremstyle]{Problem}{Problem}
\newenvironment{Solution}{\textbf{Solution.}}

\definecolor{codegreen}{rgb}{0,0.6,0}
\definecolor{codegray}{rgb}{0.5,0.5,0.5}
\definecolor{codepurple}{rgb}{0.58,0,0.82}
\definecolor{backcolour}{rgb}{0.95,0.95,0.92}

\lstdefinestyle{mystyle}{
    backgroundcolor=\color{backcolour},   
    commentstyle=\color{codegreen},
    keywordstyle=\color{magenta},
    numberstyle=\tiny\color{codegray},
    stringstyle=\color{codepurple},
    basicstyle=\ttfamily\footnotesize,
    breakatwhitespace=false,         
    breaklines=true,                 
    captionpos=b,                    
    keepspaces=true,                 
    numbers=left,                    
    numbersep=5pt,                  
    showspaces=false,                
    showstringspaces=false,
    showtabs=false,                  
    tabsize=2
}

\lstset{style=mystyle}
\newcommand{\norm}[1]{\left\lVert#1\right\rVert}     
\newcommand\course{Analysis I}                            
\newcommand\hwnumber{MATH40002}                                  
\pagestyle{fancy}
\headheight 35pt
\lhead{\today}
\rhead{\includegraphics[width=2.5cm]{icl_logo.png}}
\lfoot{}
\pagenumbering{arabic}
\cfoot{\small\thepage}
\rfoot{}
\headsep 1.2em
\renewcommand{\baselinestretch}{1.25}
\renewcommand{\labelenumi}{\alph{enumi}}
\newcommand{\Z}{\mathbb Z}
\newcommand{\R}{\mathbb R}
\newcommand{\Q}{\mathbb Q}
\newcommand{\NN}{\mathbb N}
\newcommand{\PP}{\mathbb P}
\DeclareMathOperator{\Mod}{Mod} 
\renewcommand\lstlistingname{Algorithm}
\renewcommand\lstlistlistingname{Algorithms}
\def\lstlistingautorefname{Alg.}
\newtheorem*{theorem}{Theorem}
\newtheorem*{lemma}{Lemma}
\newtheorem{case}{Case}
\newcommand{\assign}{:=}
\newcommand{\infixiff}{\text{ iff }}
\newcommand{\nobracket}{}
\newcommand{\backassign}{=:}
\newcommand{\tmmathbf}[1]{\ensuremath{\boldsymbol{#1}}}
\newcommand{\tmop}[1]{\ensuremath{\operatorname{#1}}}
\newcommand{\tmtextbf}[1]{\text{{\bfseries{#1}}}}
\newcommand{\tmtextit}[1]{\text{{\itshape{#1}}}}

\newenvironment{itemizedot}{\begin{itemize} \renewcommand{\labelitemi}{$$\bullet$$}\renewcommand{\labelitemii}{$$\bullet$$}\renewcommand{\labelitemiii}{$$\bullet$$}\renewcommand{\labelitemiv}{$$\bullet$$}}{\end{itemize}}
\catcode`\<=\active \def<{
\fontencoding{T1}\selectfont\symbol{60}\fontencoding{\encodingdefault}}
\catcode`\>=\active \def>{
\fontencoding{T1}\selectfont\symbol{62}\fontencoding{\encodingdefault}}
\catcode`\<=\active \def<{
\fontencoding{T1}\selectfont\symbol{60}\fontencoding{\encodingdefault}}
\begin{document}

\begin{titlepage}
    \begin{center}
        \vspace*{3cm}
            
        \Huge
        \textbf{
            Coursework II}
            
            
        \vspace{1.5cm}
        \Large
            
        \textbf{
        CID number: You could never find it, ahhah}% <-- author
        
            
        \vfill
        
MATH40002: Analysis I
        \vspace{1cm}
            
        \includegraphics[width=0.4\textwidth]{icl_logo.png}
        \\
        
        \Large
        
        \today
            
    \end{center}
\end{titlepage}


\newpage
\begin{Problem}
    1. This question is shamelessly stolen from Example 6.2.9(b) of "Introduction to Real Analysis" by R. Bartle and D. Sherbert.
\begin{itemize}
    \item [(a)] Prove that if $$c \in(100,105)$$ then $$10<\sqrt{c}<11$$.
    \item [(b)] Use the Mean Value Theorem to show that


    $$$$
    \frac{5}{22}<\sqrt{105}-10<\frac{1}{4}
    $$$$
    
    \item [(c)] Can you improve this estimate?
\end{itemize}  
\end{Problem}
\begin{Solution}
    \begin{proof}
        \begin{itemize}
            \item[(a)] 
            \begin{proof}
            We consider the function $$f(x)=\sqrt{x}$$ for $$x \in \mathbb{R}^+$$. We konw that $$f(x)$$ is differentiable for $$x>0$$ from the lecture. The derivative of $$f(x)$$ is $$f'(x)=\frac{1}{2\sqrt{x}}$$, which is positive for all $$x \in \mathbb{R}^+$$. Hence, $$f(x)$$ is strictly monotone incresing for $$x \in (99,122)$$. Therefore, for $$c \in (100,105)$$, $$\sqrt{100}<\sqrt{c}<\sqrt{105}<\sqrt{121}$$, which shows that
            \begin{equation*}
                10<\sqrt{c}<11
            \end{equation*}
            \end{proof}
            \item[(b)] We still consider the function $$f(x) = \sqrt{x}$$ for $$x \in \mathbb{R}^+$$. The Mean Value Theorem states that:\\
                Let $$f:[a,b] \mapsto \mathbb{R}$$ be continous on $$[a,b]$$ and differentiable on $$(a,b)$$. Then, there is $$c \in (a,b)$$ such that $$f'(c)=\frac{f(b)-f(a)}{b-a}$$.\\
            We apply the Mean Value Theorem with $$f(x)=\sqrt{x}$$ and set $$a = 100,\;b=105$$. From the lectre, we know that $$f(x)$$ is continous and differentiable for $$x>0$$, $$f(x)$$ is continous on $$[100,105]$$ and differentiable on $$(100,105)$$. The derivative of $$f(x)$$ is $$f'(x) = \frac{1}{2\sqrt{x}}$$ for $$x>0$$. Hence, by the Mean Value Theorem, we abtain that there exists a $$c \in (100,105)$$ such that
            \begin{align*}
                \frac{1}{2\sqrt{c}}&=\frac{\sqrt{105}-\sqrt{100}}{105-100}\\
                \sqrt{105}-\sqrt{100}&=\frac{5}{2\sqrt{c}}\\
                \sqrt{105}-10&=\frac{5}{2\sqrt{c}}
            \end{align*}
            By $$(a)$$, we know that $$10<\sqrt{c}<11$$, we can assert that
            \begin{align*}
                \frac{5}{2\times 11}<&\sqrt{105}-10<\frac{5}{2\times 10}\\
                \frac{5}{22}<&\sqrt{105}-10<\frac{1}{4}
            \end{align*}
            by the property of the inequality.
            \item[(c)] We can improve the estimate by using our conclusiton of (b). As $$\frac{5}{22}<\sqrt{105}-10<\frac{1}{4}$$, it follows that $$10.2272<\sqrt{105}<10.2500$$. Since $$c \in (100,105)$$, $$\sqrt{c}<\sqrt{105}<10.2500$$. Thus, we can easyily get that
            \begin{equation*}
                \sqrt{105}-10 = \frac{5}{2\sqrt{c}} > \frac{5}{2\times 10.2500}>0.2439
            \end{equation*}
            by the property of the inequality.\\
            The improved estimate is $$0.2439<\sqrt{105}-10<0.2500.$$
        \end{itemize}
    \end{proof}
\end{Solution}
\end{document}